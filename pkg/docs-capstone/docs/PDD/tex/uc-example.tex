% This is probably the most involved one because it uses the custom commands
% from use-case-table.tex.
%
% You can take a look if you want but these should abstract most of the
% low-level LaTeX stuff LaTeX commands can have zero or more arguments.Each
% argument is wrapped inside {} (squirly braces).
%
% \usecase creates a Use Case Table. It takes in two arguments:
% #1 - Table body: these should be filled with \ucrow instances.
% #2 - Table caption
%
% \ucrow creates a Use Case Table row. It also takes in two arguments:
% #1 - The left cell content. In this case it should be the detail name
% #1 - The right cell content. In this case it should be the description
%
% \ucindent simply adds a horizontal indent. It takes no arguments.
%
% Unfortunately, because this is a table, you need to use the \newline command
% to add newlines
\usecase{
	\ucrow{System}{\tpl{same as the system boundary name}}

	\ucrow{Actors}{%
		1. \tpl{all actors associated with the system} \newline%
		2. \newline%
		3.%
	}

	\ucrow{Use Case Goal}{}

	\ucrow{Primary Actor}{}

	\ucrow{Preconditions}{}

	\ucrow{Postconditions}{}

	\ucrow{Basic Flow}{%
		1. \newline%
		2. \newline%
		3.%
	}

	\ucrow{Alternate Flows}{%
		A. \newline%
		\hspace*{24pt} 1. \newline%
		\hspace*{24pt} 2. \newline%
		\hspace*{24pt} 3. \newline%
		B. \newline%
		C.%
	}

}{Use Case}{<use case name, same as the descriptor in the UML oval>}

% vim:ft=tex ts=2 sts=2 sw=2
